\documentclass[12pt,a4paper,parskip]{scrartcl}
\usepackage[utf8]{inputenc}
\usepackage[T1]{fontenc}
\usepackage[ngerman]{babel}
\usepackage{lmodern}
\usepackage[babel,german=guillemets]{csquotes}
\usepackage[style=verbose-ibid,backend=bibtex8]{biblatex}
\bibliography{baclit150214}
\usepackage{amsmath}
\usepackage{amsfonts}
\usepackage{amssymb}
\usepackage{makeidx}
\usepackage{graphicx}
\usepackage{url}
\usepackage[locale=DE]{siunitx}%SI Einheiten etc.
\usepackage[german]{fancyref}%möglicherweise rausnehmen oder justieren
\usepackage{booktabs} %Tabellen Horizontale Linier dick darstellen
\usepackage{rotating}
\usepackage{lscape}
\usepackage{subfig}
\usepackage[left=3cm,right=3cm,top=2cm,bottom=2cm]{geometry}
\begin{document}
\author{Benedikt Kaffanke}
\title{Einfluss des Ausgangs- und  Werkzeugmaterials auf Umformprozesse zur Herstellung von Verzierungselementen in der Automobilbranche}
\maketitle
\newpage
\tableofcontents
\newpage
\section{Einleitung}
In der modernen Automobilindustrie werden heutzutage immer höhere Qualitäts- und Präzisionsansprüche an die einzelnen Fahrzeugkomponenten  gestellt. So unterliegen selbst Verzierungselemente strengen Maß- und Toleranzvorgaben von Seiten der Hersteller an die Komponenten Zulieferer.\\
 Die Fertigungsprozesse solcher Präzisionsfabrikate erfordern ein hohes Maß an Überwachung und Kontrolle auf den einzelnen Fertigungsstufen. Es kommen überwiegend modernste Fertigungstechnologien (CNC-Maschinen, Industrie Roboter) zum Einsatz. Trotz hohem Automatisierungsgrad sind immer noch humane Fertigungskräfte unverzichtbar. So ist zum Beispiel bei einer \emph{Sichtprüfung} zur Verifikation der erforderlichen Oberflächengüte, des bearbeiteten Materials,  das menschliche Auge unersetzlich. Auch das Handling bei Nacharbeitungsverfahren (z.B. Polieren, Schleifen) geschieht häufig noch manuell. So  erstreckt sich das Spektrum der am Fertigungsprozess Involvierten von der einfachen Hilfskraft bis zum hochqualifizierten CNC-Spezialisten.\\
  Hinter diesem Background ist es nicht zu vermeiden das eine komplexe Anzahl von Einflussgrößen bei der Wertschöpfung als Störfaktor berücksichtigt werden müssen. Eine besondere und stetige Observation, insbesondere bei der Herstellung von sehr großen Stückzahlen,  des kontinuierlichen Flusses der Bearbeitungsschritte und der Synergie der einzelnen Elemente  der Fertigungskette, ist daher ein wichtiger Punkt zur Prävention eventueller negativer Störfaktoren. Muss zum Beispiel eine Bearbeitungsstufe, während einer Serienfertigung an einer CNC-Einheit,  aufgrund von inhomogenen Spannungsverläufen im Ausgangsmaterial häufig unterbrochen werden um Justierungen an dem Gerät durch   qualifizierte Spezialisten vorzunehmen,  ist der Kosten- und Zeitaufwand wirtschaftlich nicht mehr vertretbar.\\
     Im Fokus dieser Forschungsarbeit steht deshalb die Problematik der Optimierung der Fertigungsverfahren zur Erlangung höherer Güte bei der Herstellung von Zierleisten.
 

Zum größten Teil werden für eben diese Verzierungselemente
Strangpressprofile aus Aluminium verwendet die ein besonders hochwertiges Finish verbürgen. Sie werden in speziellen Biege- und Abkantvorrichtungen in Serie gefertigt.
 Weitere Bearbeitungsprozesse sind: \begin{itemize}
 \item Fräsen
 \item Beschneiden
 \item Schleifen und Polieren
 \item Eloxieren
 \item DURAPro Beschichten (Nanolack)
 \item Montage
 \end{itemize}
   Besondere Schwierigkeiten treten im Bereich der Maßtoleranz Einhaltung bei diesen Biegeprozessen auf. Häufig sind bei  Biegeradien und langen Profilen Toleranzen von $\pm \SI{0.5}{\milli\meter}$ gefordert. Bei kleinen Biegeradien die größtenteils bei Abkantprozessen anfallen treten optische Merkmale und Veränderungen auf, die meistens unerwünscht sind.\\
 Die Beschaffenheit des Werkstoff- und Werkzeugmaterials ist der wohl wichtigste Beeinflussungsfaktor bei o.g. Problemprodukten (siehe \fref{fig:Verdeckkastendeckel}) .
 \begin{figure}
 \centering
 \includegraphics[width=0.8\textwidth]{ZierleisteVerdeckklappendeckel}
 \caption{Problemprodukte Zierleisten und Verdeckkastendeckel}
 \label{fig:Verdeckkastendeckel}
 \end{figure}

Die nächsten Abschnitte befassen sich mit der Durchführung und Auswertung von Versuchsreihen die mit Hilfe von Messungen, herkömmlicher sowie zukunftsweisender Art (FEM-Verfahren), Erkenntnisse liefern  die die Herstellungsverfahren von Zierleisten  in qualitativer- sowie ökonomischer Sicht  optimieren.\\
Zur Untersuchung   sind hier vor allen Dingen die Umformverfahren  Kröpfen (siehe \fref{sec:kropf})  und Streckbiegen herangezogen worden. 







  
\newpage
\section{Bauteil}
Prüfobjekt ist in den folgenden Untersuchungen der Verdeckkastendeckel  des Audi A3 Cabriolet's (siehe \fref{fig:audia3}).
 Als Verzierung eines Luxusobjektes sind die Anforderungen an Aussehen und Qualität außergewöhnlich hoch. So dient er zum einen als rein optisches Veredelungselement zum anderen hat er auch funktionelle Aufgaben (z.B. Stabilität in den gesamten Kofferraumdeckel bringen oder auch als Antenne zu agieren). Geringe Spaltmaße,  perfekte Symmetrie (das menschliche Auge erkennt ein Hundertstel Millimeter) so wie allgemeine Benutzerfreundlichkeit (z.B. Hängenbleiben von Kleidungsstücken und Ähnlichem an dem Verzierungsobjekt sollte ausgeschlossen sein) sind Anforderungen die höchste Priorität haben.
 Darüber hinaus sind  flüssige Übergänge und Einklang   zu weiteren Verzierungselementen des Fahrzeuges von großer Bedeutung für einen harmonischen Gesamteindruck.
 
   
\begin{figure}
\centering
\hfill
\subfloat[Audi A3 ]{\includegraphics[scale=.4667]{audia3blau}}
\hfill
\subfloat[Audi A3 Verdeckkastendeckel \label{fig:audia3verdeck}]{\includegraphics[scale=.263]{audivdkd}}
\hfill
\caption[Audi A3 Endprodukt]{Audi A3 Endprodukt\footnotemark }
\label{fig:audia3}
\end{figure}
 \footnotetext{Vgl.http://www.cars.co.za/motoring\_news/2014-audi-a3-cabriolet-completes-the-a3-family/6061/[28.12.2013].}



	 	 
\subsection{Funktion \& Qualitätsumfang}
An Verzierungselemente werden gerade in der Automobil Oberklasse besonders hohe Ansprüche gestellt. Es sind besonders folgende hervorzuheben:
\begin{itemize}
\item keine Beulen
\item keine Oberflächenfehler
\item ideale Fugenläufe
\item präzise Radien
\item enge Form- und Lagetoleranzen (siehe \fref{fig:vdkdtol})
\item enge Spalttoleranzen
\end{itemize}
\begin{figure}
  \centering
  \includegraphics[width=0.8\textwidth]{vdkdtol}
  \caption{Wölbungstoleranz}
  \label{fig:vdkdtol}
  \end{figure}






\subsection{Aluminium}
Aufgrund seiner geringen Dichte (\SI{2.69}{\kilo\gram\per\deci\meter\cubed})\footcite[Vgl.][353]{wm}, guten Umformbarkeit, Korrosionsbeständig und mit einer hervorragend zu erzielender Oberflächengüte sowie hohem Reflexionsgrad ist Aluminium das am häufigsten verwendete Ausgangsmaterial für Zierleisten.\\
 Es werden überwiegend Strangpressprofile verarbeitet die bei den Lieferanten mit bestimmten Eigenschaften angefordert werden. Die wichtigsten dort angeführten mechanischen Eigenschaften sind die Zugfestigkeit $R_m  [\si{\newton\per\milli\meter\squared}$], Dehnung $R_{po,2} [\si{\newton\per\milli\meter\squared}]$,   Bruchdehnung A oder auch $A_{50} [\si{\percent}]$ (der Index 50 bezieht sich auf eine Messlänge  von \SI{50}{\milli\meter} der Probe  beim einachsigen Zugversuch)\footcite[Vgl.][281]{aa}und die Korngröße.\\
  Sie wird in der Einheit [\si{\micro\meter\squared}] angegeben und hat Einfluss auf die Oberflächengüte nach  Umformprozessen. Bei steigendem Umformgrad ergibt sich häufig eine Aufrauung der Oberfläche (Orangenhaut)\label{sec:orangenhaut} die von der Ausgangskorngröße abhängig ist. Je geringer die Ausgangskorngröße desto geringer der Aufrauungseffekt.\footcite[Vgl.][524]{aa}\\ Stark verformtes und grobkörniges Material entwickelt oft in den deformierten Zonen (insbesondere in den gestreckten Bereichen) eine Oberflächenrauigkeit (Orangenhaut), die die Reflektivität und Einfärbbarkeit des Endproduktes stark einschränkt. Das Phänomen \emph{Orangenhaut} entsteht vorwiegend an Umformbereichen die nicht in direktem Kontakt mit Werkzeugoberflächen stehen.\footcite[Vgl.][19]{hmp}
Erwähnenswert ist zu Vorangegangenem noch, dass aufgrund der bei den meisten Aluminiumlegierungen, nicht ausgeprägten Streckgrenze  die $R_{p0,2}$ Dehngrenze als Bemessungskennwert bei einer \SI{0.2}{\percent} bleibenden Verformung gegenüber rein elastischem Verhalten ermittelt wird (siehe \fref{fig:spanndehn2}).\footcite[Vgl.][280-281]{aa}  
\begin{figure}
\centering
 	\includegraphics[width=0.8\textwidth]{spanndehn2}
 	\caption{Spannungs-Dehnungs Schaubild mit $Rp_{0,2} $ Dehngrenze}
 	\label{fig:spanndehn2}
 	\end{figure}
	 	 	 	 
	 	 	 	 

\subsection{Streckbiegen}
Unter Biegen versteht man nach DIN 8586 das Umformen von festen Körpern, wobei der plastische Zustand im Wesentlichen durch eine Biegebeanspruchung herbeigeführt wird.\footcite[Vgl.][376]{hu} Es wir bei dem Verfahren \emph{Streckbiegen} bei Raumtemperatur (\SI{20}{\degreeCelsius})  geformt, daher fällt es in die Rubrik \emph{Kaltumformen} (DIN 8582). Da bei Raumtemperatur ein begrenztes Formänderungsvermögen vorliegt sind höhere Umformkräfte erforderlich. Der Vorteil des Kaltumformens ist eine hohe Maßgenauigkeit.\footcite[Vgl.][8]{hu}. Eine weitere Definition besagt, dass Kaltumformen vorherrscht solange die Umformtemperatur des Werkstoffs geringer als seine Rekristallisationstemperatur ist.\footcite[Vgl.][187]{fu} Die Blechumformung verfolgt generell das Ziel aus einem Flachprodukt ein räumliches Gebilde zu formen, ohne dabei (im Idealfall) die Blechdicke zu verändern.  Eine Formänderung vollzieht sich aus diesem Grunde hauptsächlich in der Blechebene unter ebenem Spannungszustand. Als Grundverfahren in der Blechumformung sind Verfahren wie Tiefziehen, Biegen und Streckziehen (oder auch Streckbiegen) zu nennen. Gemeinsam haben sie alle, dass sich Stauch- und Streckverformungen in der Blechebene und Blechdicke abspielen und unterschiedliche Dehnungszustände und -abläufe anzutreffen sind.\\ Die Umformbarkeit (Duktilität) als Werkstoffeigenschaft ist wegen der Tatsache, dass  Spannungs- und Dehnungszustände mit den Fließ- und Brucheigenschaften eines Werkstoffes in Wechselwirkung stehen, ein komplexes Gebiet. Über das Werkzeugsystem (Stempel, Niederhalter etc.) werden die benötigten Umformkräfte in das Werkstück eingeleitet und erwirken so über die  Formänderungen, Relativbewegungen zwischen Werkzeug und Werkstück bei veränderlichen Anpreßkräften. Die zwischen Bauteil (Werkstoff) und Wirkteil (Werkzeug z.B. Stempel) entstehenden Reibungsverhältnisse resultieren aus den  Grenzflächen- und Gleiteigenschaften von Wirkteil und Werkstoff.\\ Insbesondere bei Aluminiumwerkstoffen haben sie bedeutenden Einfluss auf das Umformergebnis. Der Werkzeugaufbau und die Werkzeuggeometrien so wie die Steuerung des Fertigungsvorgangs haben erheblichen Anteil an der Art und Weise des Werkstoffflusses bei den Biegeprozessen.\footcite[Vgl.][499]{aa}
Bei dem Umformverfahren Streckbiegen werden auf speziellen Streckbiegemaschinen die Enden eines Profilstranges in Spannern gehalten und auf  Zugspannung gebracht (siehe  \fref{fig:Streckbiegemaschine}). Anschließend werden sie über ein massives Biegewerkzeug streckgebogen.\footnote{Vgl.\url{http://www.tillmann-gruppe.de/de/streckbiegen.html}[27.10.2013].} 
\begin{figure}
\centering
\includegraphics[width=0.8\textwidth]{Streckbiegemaschine}
\caption{Streckbiegemaschine}
\label{fig:Streckbiegemaschine}
\end{figure}
Das Streckbiegeverfahren was bei der Firma DURA  für den Verdeckkastendeckel des Audi A3 eingesetzt wird ähnelt mehr dem \emph{Tangentialstreckziehen}. Der Unterschied zu dem herkömmlichen Streckziehen das in einer Arbeitsstufe erfolgt und bei dem die Zugspannung nur über den Stempel eingeleitet wird ist das Fertigen in zwei Arbeitsschritten.

 Im ersten Schritt wird das Strangpressprofil in die Spannvorrichtung der Steckbiegemaschine eingelegt und an den Enden eingespannt. Danach fahren die Spannelemente horizontal auseinander und leiten eine Zugspannung in das Bauteil ein. Es wird knapp über den Bereich der Streckgrenze gestreckt. Je nach Material und erwünschtem Biegeresultats werden die aufgewandten Zugkräfte von den Maschineneinrichtern präzise eingestellt.\\
  Im zweiten Schritt erfolgt nun die eigentliche Formgebung. Das gestreckte Strangpressprofil wir unter Aufrechterhaltung der eingebrachten Zugspannung mit einer kontinuierlichen Geschwindigkeit tangential um das formgebende Wirkteil gelegt. Die Bewegung wird von den Spannelementen alleinig ausgeführt (siehe \fref{fig:streckbiegen})\footnote{\url{http://www.custompartnet.com/wu/sheet-metal-forming}[04.02.2014].}. 
  \begin{figure}
  \centering
  \includegraphics[width=0.8\textwidth]{streckbiegen}
  \caption{Prinzip Streckbiegen}
  \label{fig:streckbiegen} 
  \end{figure}
  
Das Ausgangsmaterial (Aluminium Strangpressprofile) wird streckgebogen um eine Rückfederung (siehe \fref{fig:springback})\footcite[Vgl.][65]{smfpd} zu minimieren.  Die Rückfederung entsteht aus der Rückbildung der elastischen Formänderung des Bauteils nach der Beendigung des Biegevorgangs und nach dem Entfernen der eingeleiteten Kräfte. Es findet im Werkstoff eine elastische Erholung statt. In der Blechmanufaktur und Blechumformung stellt die Rückfederung eine der bedeutendsten Formgebungsproblematiken dar.
\\
 Sie ist ein äußerst komplexes Phänomen da sie von mannigfaltigen Interaktionen zwischen den Materialeingenschaften, der Bauteilgeometrie, der Reibung,  der Werkzeugradien und weiteren formgebenden Bedingungen abhängt. Unter den Variablen, die die Rückfederung verringern sind der Reibungskoeffizient und die Reibung, die Streckkraft, die Nachbiegekraft, die Formänderungsgeschwindigkeit, die Temperatur sowie weitere geometrische Parameter zu bezeichnen. Hervorzuheben ist an dieser Stelle, dass das Verhältnis des Biegeradius $ R $ zur Blechdicke $ t $ in einem Bereich von $ \frac{R}{t} < 10 $ die Rückfederung wesentlich vergrößert. Im laufe der Zeit haben sich verschiedenen Methoden zur Unterdrückung der Rückfederung bewährt. Zu ihnen zählen das Überbiegen, das Strecken und das Nachdrücken oder Nachpressen. Zur Kompensation der Rückfederung hat sich das Überbiegen als das sicherste Verfahren in der Vergangenheit erwiesen. Es wird in der Fertigungspraxis davon ausgegangen das ein Überbiegungsspielraum von 2 \%  bei Stahlbauteilen ausreichend ist um Rückfederungseffekte zu minimieren.\\ Das Nachpressen in der Biegeregion hat den Nachteil, dass hohe Presskräfte aufgebracht werden müssen um einen gewünschten Effekt zu erzielen. Bei dem Verfahren  Streckbiegen zur Optimierung der Rückfederung wird das Bauteilmaterial zu erst über den Bereich der Streckgrenze hinweg (meistens durch Aufbringung hydraulischer Zugkräfte) gestreckt und dann über das formgebende Werkzeug gebogen. Dieses Verfahren wird nur bei großen Biegeradien angewandt weil kleine Radien eine sehr große Vorspannung (jenseits der Mindestzugfestigkeit) benötigen würden.\footcite[Vgl.][16-19]{hmp}\\  \begin{figure}
 \centering
 \includegraphics[width=0.8\textwidth]{springback}
 \caption{Prinzip der Rückfederung}
 \label{fig:springback}
 \end{figure}
Ein weiterer kritischer Punkt bei dem Verfahren Streckbiegen und Biegeprozessen generell sind die Spannungsverläufe in der Verformungszone die Eigenspannungen in das Bauteil bringen welche bei späteren Bearbeitungsverfahren wie z.B. das Beschneiden oder Fräsen in der Nähe des Deformierungsbereiches Verzug und Maßänderungen hervorrufen.\\ Je nach dem wie weit das Ausgangsmaterial über die Streckgrenze hinaus getreckt wird (bis zu welchem Grad das Material fließt) verschiebt sich die neutrale Faser in dem Biegebereich in Richtung des kleinen Radius (also nach innen). Bei sehr großer Streckung ist es möglich das die neutrale Faser sogar außerhalb des Bauteils liegt.\footcite[Vgl.][374]{fu} Bei den Versuchsreihen welche in dieser Ausarbeitung durchgeführt werden, bleibt die neutrale Faser innerhalb des Bauteils weil nur sehr knapp über die Streckgrenze hinaus gestreckt wird. Der typische Spannungsverlauf resultiert daher in Zugspannungen im Aussenbereich der Biegeradien und Druckspannungen in dem Innenbereich (siehe \fref{fig:neutralefaser})\footcite[Vgl.][195]{tsch}.
\begin{figure}
\centering
\includegraphics[width=0.8\textwidth]{neutralefaser}
\caption{Spannungen in der Deformierungszone beim Streckbiegen}
\label{fig:neutralefaser}
\end{figure}
In Hinblick auf die Serienfertigung sind die in den für den Fertigungsprozess  Streckbiegen durchzuführenden Versuchsserien, bei denen unter Verwendung von Strangpressprofilen der Verdeckkastendeckel des Audi A3 Cabriolet's gefertigt wird, spezifische Problembereiche besonders zu beachten.\\
Kritisch sind hier vor allen Dingen Biegeschwankungen und nicht kontinuierliche Materialeinschnürungen,  welche häufig an den Verengungen der Biegeradien auftreten. Die einflussreichsten mechanischen Eigenschaften des Werkstoffes sind bei diesem Verfahren die Härte sowie die Streckgrenze.

\subsection{Kröpfen \label{sec:kropf}}
Der eigenartig anmutende Ausdruck \emph{Kröpfen} bedeutet eigentlich nur \emph{krumm biegen}.\footnote{Vgl.\url{
http://woerterbuchnetz.de/DWB/?sigle=DWB&mode=Gliederung&lemid=GK14769
}[27.10.2013].}
Bei dem Umformprozess Kröpfen werden von den  Enden der Zierleisten zu nächst die auf den Innenseiten verlaufenden Stege  abgefräst.  Daraufhin werden sie in der Kröpfeinheit (siehe \fref{fig:kropfeinheit}) auf dem Kröpfstein justiert und von einem Niederhalter durch die Anpresskraft einer Gasdruckfeder angepresst. Nun fährt, angetrieben durch einen Hydraulikzylinder, der Kröpf- oder auch Ziehstempel herunter und kantet das Material ab. Im Anschluss daran wird die Stirnseite der Kröpfung (siehe \fref{fig:kropfinstirn}) noch beschnitten.
\begin{figure}
\centering
\hfill
\subfloat[Bezeichnungen \label{fig:kropfeinhzeich}]{\includegraphics[scale=.715]{kropfeinhzeich}}
\hfill
\subfloat[Kröpfstein\label{fig:einheit}]{\includegraphics[scale=.5]{kropfeinheit}}
\hfill
\caption{Kröpfeinheit }
\label{fig:kropfeinheit}
\end{figure}

\begin{figure}
\centering
\hfill
\subfloat[Kröpfung mit Fräsbereich  \label{fig:kropffrasbereich}]{\includegraphics[scale=.822]{kropffrasbereich}}
\hfill
\subfloat[Kröpfung \label{fig:kropfung}]{\includegraphics[scale=.16]{kropfung}}
\hfill
\caption{Kröpfung innen und Stirnseite }
\label{fig:kropfinstirn}
\end{figure}


 
Problembereiche sind hier zu erst einmal die Fräsprozesse. Schon bei geringsten Unterschieden in der Materialabnahme sind Fehlstellen in der Oberflächenqualität der Radien bei einer Sichtprüfung zu erkennen. Auch der Ziehstempel und der Kröpfstein lassen Spuren auf der Oberfläche zurück. Ein nicht zu vernachlässigender Aspekt ist auch der Verschleiß des Werkzeugmaterials bei diesem Verfahren. So kommt es gerade bei Ziehstempeln aus Stahl oft zu Kaltaufschweißungen. Hier liegt nahe auch andere Werkzeugmaterialien in Versuchsreihen zu erproben.
\medskip

Hervorzuheben sind folgende, aus dem Kröpfprozess resultierende, Qualitätsbeeinträchtigungen:
\begin{itemize}
\item Orangenhaut (siehe \fref{sec:orangenhaut})
\item Materialungänzen bedingt durch Materialschwankungen
\item Abweichungen des auf das Kröpfen angepassten Fräsbildes
\end{itemize}

Das Fertigungsprinzip Kröpfen  ist bei der Firm DURA im  Laufe der Jahre immer weiter Optimiert worden. Die mechanischen Vorgänge sind für einen Aussenstehenden aufgrund der kompakten Bauweise der Kröpfheinheiten zunächst schwer zu durchschauen. Eine sehr vereinfachte Prinzipdarstellung ist zum Verständnis sehr hilfreich (siehe \fref{fig:krpfgesamt}).\\
 Es ist in dem Vertikalschnitt deutlich zu sehen, dass das Werkstoffmaterial (rot) nach unten \emph{herausgekämmt} wird. In dem Horizontalschnitt erkennt man deutlich das der Werkstoff (rot) in der Kavität des Ziehstempels geführt wird und ein seitliches Ausbrechen nicht möglich ist.\\ In der Sequenz (siehe \fref{fig:krpfprinz}) wird der ganz Vorgang noch einmal transparent verbildlicht. Das Ausgangsmaterial wird zunächst auf den \emph{Kröpfstein} (blau) gelegt danach wird es durch einen Niederhalter (gün) fixiert bevor der \emph{Ziehstempel} herunterfährt und das Bauteil verformt. \\    Der Ziehstempel schlägt zuerst von oben  (a) auf das Bauteil  und knickt (oder biegt) das Bauteil in den ersten Umformgraden um. Da der Ziehspalt (Abstand zwischen Kröpfstein und Innenwand des Ziehstempels) schmaler als das Bauteil ist, erfolgt nahezu gleichzeitig  ein Kaltumformprozess der dem Drahtziehen am ähnlichsten ist (im Bild mittlere Einstellung (b)). \\ Nach Beendigung des Umformprozesses (c) ist die abgekantete Seite natürlich flacher und länger aufgrund der \emph{Volumenkonstanz}. Der ganze Vorgang ist somit eine Symbiose aus dem \emph{freien Biegen}, welches den Radius verursacht und einer Variante des Ziehens (oder auch Drahtziehens) bei der ein Fließen des Werkstoffes auftritt und welches auch eine Kaltverfestigung mit sich bringt. \\
Eine Analogie zu dem Umformverfahren Drahtziehen und den dortigen Spannungsverläufen in der Deformierungszone (siehe \fref{fig:eigenspandrahtzieh}) bietet sich an, weil bei dem Kröpfprozess sowie bei dem Drahtziehprozess der Werkstoff in die gleiche Richtung wie die relative Bewegungsrichtung des  formgebenden Werkzeugs (bei dem Drahtziehen die Matrize und bei dem Kröpfen der Ziehstempel) fließt. Bei beiden Verfahren findet ein \emph{Herauskämmen} des Materials aus der Umformzone statt.
\begin{figure}
\centering
\includegraphics[width=0.8\textwidth]{krpfgesamt.png}
\caption{Kröpfen prinzipiell mit Schnitten in der Umformzone}
\label{fig:krpfgesamt}
\end{figure}

\begin{figure}
\centering
\includegraphics[width=0.8\textwidth]{krpfsequenz}
\caption{Sequenz des Umformvorganges Kröpfen}
\label{fig:krpfprinz}
\end{figure}



  











\subsection{Methode Messauswertung}
Es wurde bei der Auswertung von Messreihen in dieser Untersuchung vorwiegend die \textbf{\emph{empirische}} Standardabweichung 
\begin{equation}s= \sqrt{\frac{\sum \limits_{i=1}^n (x_i - \bar{x})^2}{n-1}}\end{equation}  verwendet, welche für solche Operationen von der Fachliteratur empfohlen wird.\footcite[Vgl.][301]{mf} Der Unterschied zur Standardabweichung \begin{equation} \sigma = \sqrt{\frac{\sum \limits_{i=1}^n (x_i - \bar{x})^2}{n}}\end{equation}  ist das \emph{Teilen} durch \textbf{n-1} anstatt durch lediglich \textbf{n}.
 An dieser Stelle eine kurze Beleuchtung des Sachverhaltes\footnote{\url{www.ti-unterrichtsmaterialien.net/imgserv.php?id=pinkernell_106.pdf}[10.11.2013].‎}.



Die \textbf{\emph{empirische}} Standardabweichung berechnet das Streuungsmaß einer \emph{Stichprobe} im Gegensatz zur Standardabweichung die sich auf eine \emph{Grundgesamtheit} bezieht. Bei Stichproben wir die \emph{empirische} Standardabweichung vorgezogen da dort in der Regel die \emph{wirkliche Streuung} unterschätzt wird. Die \emph{empirische} Standardabweichung ist wegen des Teilers n-1 grundsätzlich etwas größer als die Standardabweichung, bei großem n liefern aber beide nahezu gleiche Ergebnisse, welches ja nur eine logische Konsequenz ist, denn je größer die Stichprobe desto näher kommt sie an die Grundgesamtheit.

\medskip

Durch das Quadrieren der einzelnen Abweichungen ($ x_i-\bar{x}$) und Addieren der einzelnen Abweichungsquadrate erhält man nur positive Beträge in denen eine Überbetonung einzelner Ausreißer erzielt wird.
Die empirische Standardabweichung ist   eines der wichtigsten Vergleichsparameter in der Statistik und bietet sich zur Analyse der Versuchsreihen besonders an, da sie von Extremwerten nicht stark beeinflusst wird.\footcite[Vgl.][54]{gst} Bei der Auswertung von Messbereichen, die für unsere Problemstellung besondere Signifikanz haben, wird zusätzlich der Fehler mit Hilfe der  \emph{Standardabweichung des Mittelwertes}  \begin{equation} \Delta\bar{x}= t_{0,95} \cdot \sqrt{\frac{\sum \limits_{i=1}^n (x_i - \bar{x})^2}{(n-1)\cdot n}}\end{equation}  angegeben.\footcite[Vgl.][16]{ph} Da bei den Versuchsserien eine nicht allzu große Stückzahl ($ n=16-20 $) bearbeitet wurde,  ist auch der für die international geforderte statistische Sicherheit zu berücksichtigende \emph{P} Wert mit dem $ t_{0,95} $ Faktor in die Berechnungen eingegangen.\footcite[Vgl.][609]{tp}  Es sei noch bemerkt, dass der Fehler nach DIN 1333 jeweils auf die erste signifikante Stelle gerundet wurde.\footcite[Vgl.][612]{tp}
 	 	
 	

 	
 	 	


\subsection{Chargenvergleich Streckbiegen}
Zur Versuchsdurchführung wurden drei Materialchargen zu jeweils 20 Profilen des Werkstoffes EN AW 6060 (Legierungsnummer EAL-6048 \emph{Alminox}, AlMgSi\,0,5) mit den Materialbezeichnungen F17 (T61/Charge 1) und Fxx (T4/Charge 2) sowie das ursprünglich zur Serienfertigung vorgesehene Material F13 (T4)  gegenübergestellt (eine Übersicht der relevantesten Eigenschaften ist in \fref{tab:chargeneigenschaften} aufgeführt).
\begin{table}[htbp]
\caption{Gegenüberstellung der mechanischen Eigenschaften (Laborwerte) der Chargen}
\label{tab:chargeneigenschaften}
\centering
\begin{tabular}{lllll}
\toprule
Material & Zugfestigkeit & Streckgrenze & Bruchdehnung & Zustand \\
Charge &  Rm [\si{\newton\per\milli\meter\squared}] &  $R_{p0,2}$ [\si{\newton\per\milli\meter\squared}] &  $A_{50}$ [\%] & \\
\midrule
1.F17 & 160,25 & 85,55 & $ 12,3 $ & T61 \\
2.Fxx & 152,4 & 74,65 &  $ 11,65 $ & T61 \\
3.F13 Serie & 149,3 & 70,55 & 20,41  & T4 \\
\bottomrule




\end{tabular}
\end{table}


 Die Chargen 1 und 2 wurden auch mit der herkömmlichen Zustandsbezeichnung T61 (lösungsgeglüht, nicht vollständig warmausgelagert, überaltert)\footnote{Vgl.\url{http://www.unibw.de/lrt5/lehre/praktikum/zusatzinformationen/download4/at_download/down1}[25.11.2013].} bezeichnet während das Serienmaterial im Zustand T4 (lösungsgeglüht, kaltausgelagert) bestellt wurde. \\
 Unter Überalterung versteht man  den Prozess der Vereinigung von  submikroskopischen Ausscheidungen die sich  in der Anzahl verringern jedoch als Ausscheidung größer werden und so eine Abnahme der Festigkeit herbeiführen.\footcite[Vgl.][52]{wki}\\
  Lösungsglühen erfolgt durch Glühen im Bereich der homogenen Mischkristalle welches das   Lösungsvermögen der Mischkristalle begünstigt, Ausscheidungen können so gelöst werden.\\
   Unter Auslagern versteht man Liegenlassen bei Raumtemperatur (Kaltauslagern) oder bei  höheren Temperaturen (Warmauslagern), meistens zwischen 100 und 220 Grad Celsius, über einen bestimmten Zeitraum um so die Eigenschaften des Werkstoffes zu beeinflussen.\footcite[Vgl.][213]{wk}
Ein typischer Aushärtungsprozess läuft nach folgendem Schema ab:

\begin{enumerate}
 \item Lösungsglühen aller Ausscheidungen in einem homogenen Mischkristall 
 \item Abschrecken
 \item Auslagern 
 \end{enumerate}
 
   
 
 

Die Zusstandsbezeichnungen F17, F13 und Fxx beziehen sich nach DIN 755-2 auf die Zugfestigkeit. Fxx ist allerdings eine firmeninterne Bezeichnung und bedeutet das ein  vorgezogener Kaltauslagerungsprozess durchgeführt wurde um das Strangpressprofil zu "`\emph{stabilisieren}"'. Das bedeutet ein gewisses "`Einfrieren"' des Gefüges in den momentanen Zustand um Veränderungen desselbigens auch bei nicht vorgesehener längerer Lagerung zu verhindern. Nach Auskunft des Lieferanten ist Fxx leicht wärmebehandelt worden.


 Bei Charge 2 (Fxx) schieden zwei Profile aufgrund von Biegefehlern aus. Die Proben wurden streckgebogen und auf einer Messlehre  mit 40 Messpunkten (Messpunkte MP1a bis MP10d) vermessen. Die Messbereiche, Messpunkte und Messuhren wurden, zur besseren Übersicht, mit Farben markiert (siehe \fref{fig:messpunktevdkda3}).  \\
 And den Messpunkten wurden folgende Messbereiche ermittelt:
 \begin{description}
 \item[MP1a-MP10a] Kontur aussen (grün)
 \item[MP1b-MP10b] Spalt (gelb)
 \item[MP1c-MP10c] Wölbung oben innen (rot)
 \item[MP1d-MP10d] Wölbung oben aussen (blau)
 \end{description}
\begin{figure}
\centering
\includegraphics[width=0.8\textwidth]{messpunktevdkda3}
\caption{Messpunkte Biegelehre}
\label{fig:messpunktevdkda3}
\end{figure} 
 
 
 
 
Das Messen erfolgte durch Abfahren aller Messpunkte mit den zu den spezifischen Messbereichen zu verwendenden Messuhren (siehe \fref{fig:messverfahren}). Ein negativer Messwert lässt auf eine Verkleinerung des Messbereiches schließen. Eine Ausnahme hierzu ist der Messbereich "`\emph{Spalt vorne unten}"' welcher  bei negativen Werten eine Vergrößerung bedeutet.\\
 Alle relevanten Messergebnisse (mit Ausnahme der Messpunkte MP1b und MP10b bei Charge 1, welche nicht zu ermitteln waren) wurden in Tabellen eingetragen und  der Mittelwert sowie die Standardabweichung
 ermittelt. Darüber hinaus erfolgte eine Gegenüberstellung der spezifischen Werte.\\
Da aufgrund der vielen Messpunkte  sehr umfangreiche Auswertungen durchgeführt wurden,  sind hier die für die Problematik Ausschlaggebendsten näher betrachtet worden. Alle weiteren Messergebnisse und Visualisierungen sowie Dokumentationen sind dem Anhang zugefügt. 
\begin{figure}
\centering
\hfill
\subfloat[Messlehre  \label{fig:messlehre}]{\includegraphics[scale=.0877]{Messlehre}}
\hfill
\subfloat[Messung "`Kontur aussen"' \label{fig:messvdkd}]{\includegraphics[scale=.042]{messvdkd}}
\hfill
\caption{Messverfahren an der "`Biegelehre"' }
\label{fig:messverfahren}
\end{figure}


Der für das Streckbiegen aussagekräftigste Parameter ist der Messbereich "`\emph{Kontur aussen}"' da er dem Verlauf der Biegelinie entspricht. Besonders an den Messpunkten MP1a und MP10a sind die Auswirkungen der Rückfederung zu beobachten. Ein Vergleich der Chargen ist in \fref{tab:mwertstandstreck} übersichtlich dargestellt. Ein visueller Vergleich der Standardabweichungen der "`Kontur aussen"' ist in \fref{fig:svstb} aufgeführt.
\begin{figure}
\centering
\includegraphics[width=0.8\textwidth]{standardstreckb}
\caption{Überlagerung Standardabweichungen "`Kontur aussen"' Streckbiegen}
\label{fig:svstb}
\end{figure}
Dort ist zu sehen, dass das Material F17 (Charge 1) an fast allen Messpunkten die geringste Standardabweichung aufweist. Lediglich bei Messpunkt MP3a liegt sie in nicht großem Abstand zwischen dem Fxx (Charge 2) und dem F13 (Serie) Material.

Ein Vergleich der Mittelwerte (Kontur aussen) der Chargen (siehe  \fref{fig:mitstrckb}) ergibt, dass and den Messpunkten MP1a, MP2a, MP9a und MP10a  Charge 1 (F17) die größte Rückfederung nach dem Streckbiegeprozess auftritt.






\begin{table}
\caption{Messwerte und Standardabweichungen Streckbiegen "`Kontur aussen"'} 
\label{tab:mwertstandstreck}
\vskip\abovecaptionskip



\footnotesize
   \begin{tabular}{cccccc}
   \toprule
   & \multicolumn{5}{c}{Messwert $x =  (\bar{x} \pm \Delta x)  $[mm]}\\
   \cmidrule(ll){2-6}
   Material    & MP1a & MP2a & MP3a & MP4a & MP5a \\  
   
   \midrule
  
   F17& $ 1,95\pm 0,13 $ & $0,72 \pm 0,06 $ & $0,60 \pm 0,04 $ & $ 0,028 \pm 0,017 $ & $-0,297 \pm 0,014$ \\
    Fxx & $0,81 \pm 0,21 $ & $0,34 \pm 0,07 $ & $0,15 \pm 0,05 $ & $0,021 \pm0,027 $ & $ -0,188\pm0,030$ \\
   F13  Serie & $-0,66 \pm0,22 $ & $-0,13 \pm 0,06 $ & $-0,38 \pm 0,04$ & $ 0,028\pm0,024 $&$0,00 \pm0,05 $ \\
     \bottomrule
     \toprule
  Material   & MP6a & MP7a & MP8a & MP9a & MP10a  \\
  \midrule
      F17&   $-0,368 \pm0,014 $&$-0,293 \pm0,012 $&$ 0,46 \pm 0,06 $&$ 1,31\pm 0,04$&$2,96 \pm0,10 $ \\
Fxx &$-0,233 \pm0,024 $&$-0,251 \pm0,015 $&$-0,17 \pm0,07 $&$0,57 \pm0,06 $&$1,37 \pm0,15 $ \\
  F13 Serie & $-0,04 \pm0,08 $&$ -0,16\pm0,11 $&$-0,55 \pm0,11 $&$0,04 \pm  0,10$&$-0,08 \pm0,21$ \\
     
     \bottomrule
      
     &&&&&\\
     &&&&&\\
     &&&&&\\
     &&&&&\\
     &&&&&\\
     
     \toprule
      & \multicolumn{5}{c}{Standardabweichung s [mm]}\\
   \cmidrule(ll){2-6}
   Material    & MP1a & MP2a & MP3a & MP4a & MP5a \\ 
   \midrule
    F17&0,270&0,116&0,086&0,036&0,028\\
    Fxx &0,416&0,138&0,098&0,053&0,060\\
    F13 Serie&0,454&0,121&0,068&0,051&0,103\\
     \bottomrule
     \toprule
  Material    & MP6a & MP7a & MP8a & MP9a & MP10a  \\
  \midrule
    F17 &0,028&0,025&0,113&0,078&0,210\\
 Fxx   &0,048&0,028&0,132&0,121&0,291\\
F13 Serie &0,164&0,219&0,235&0,211&0,432\\ 
   \bottomrule 
         
   \end{tabular} 
\end{table}

\begin{figure} 
\centering
\includegraphics[width=0.8\textwidth]{mitkontausstrckb}
\caption{Vergleich Mittelwerte "`Kontur aussen"' Streckbiegen}
\label{fig:mitstrckb}
\end{figure}

\newpage

\subsubsection{Ausblick}
In folge der Überlagerung der Standardabweichungen (Streckbiegen "`Kontur aussen"') der untersuchten Chargen unter dem Gesichtspunkt der Mindestzugfestigkeiten (siehe \fref{fig:standzugrel}) ist einzusehen, dass das Material F17 (Charge 1) bei einer Mindestzugfestigkeit von Rm = 160  \si{\newton\per\milli\meter\squared}  die geringste Standardabweichung hat. Bei Werten von s = (0,025  bis 0,270) \si{\milli\meter}  ist davon auszugehen das auch größere Stückzahlen mit relativ geringen Prozessschwankungen zu fertigen sind. Hier müssen jedoch eventuelle Montageprobleme des Verdeckkastendeckels aufgrund der höheren Rückfederungswerte von {$x_{\text{Rückfeder}}$ = (-0,368  bis 2,96) \si{\milli\meter} berücksichtigt werden. Eine Tatsache die bei einer Spannweite von 3,328 \si{\milli\meter} schon einen beachtlichen Spielraum beim Einbau und bei der Passform bedarf.  % montage wegen der Toleranzen fragen
Hier ist das Ausmaß von Wölbungen und Spannungen nach und während der Montage schon genau zu untersuchen.

\begin{figure}
\centering
\includegraphics[width=0.8\textwidth]{standzugrel}
\caption{Übersicht Mindestzugfestigkeit/Standardabweichung "`Kontur aussen"'}
\label{fig:standzugrel}
\end{figure} 


Unter der Voraussetzung geringer Prozessschwankungen im Streckbiegeverfahren welche bei geringer Standardabweichung unter sorgfältiger und präziser Auswahl des Vormaterials durchaus zu realisieren sind, können die Ausschussrate sowie Kosten und Zeitverluste die durch ständiges Justieren der Streckbiegemaschine durch geschultes Personal entstehen, erheblich reduziert werden.

In Anbetracht der vorangegangenen Auswertung wurden noch einmal zwei Chargen (F19 und F18) bei dem Zulieferer, zu Versuchszwecken, bestellt. Möglicherweise ist hier ein Material herauszukristallisieren welches noch geringere Prozessschwankungen ermöglicht. Wir sind dabei von einer steigenden Zugfestigkeit ausgegangen da sich nach den Diagrammen in \fref{fig:standzugrel} und \fref{fig:standstreckrel} die Standardabweichung sowie Mindestzugfestigkeit und Streckgrenze gegenläufig verhalten.


 











\newpage
\section*{Anhang}
\include{ExkursUmform}



\printbibliography

\end{document}  
